% Основные параметры
\documentclass[a4paper, 12pt]{article}
\usepackage[left=1.5cm, right=1.5cm, top=1.5cm]{geometry}
% Ссылки
\usepackage[
  colorlinks=true,
  urlcolor=blue,
  pdftitle={Резюме},
  pdfauthor={Ефим Ишенин}
]{hyperref}
% Настраиваемые списки
\usepackage{enumitem}
% Шрифт Roboto
\usepackage{fontspec}
\setmainfont{Roboto}
% Убрать нумерацию страниц
\pagenumbering{gobble}
% Убрать отступы параграфов
\setlength{\parindent}{0pt}
% Настройка заголовков
\usepackage{titlesec}
\titleformat{\section}
  {\normalfont\large\bfseries}
  {}
  {0pt}
  {}
  [\vspace{1pt}\titlerule\vspace{-6.5pt}]
% Поддержка русского языка
\usepackage[russian]{babel}

\begin{document}

% Имя и контактная информация
\centerline{\Huge Ефим Ишенин}
\vspace{10pt}
\centerline{
  Россия, Красноярск |
  \href{https://github.com/Efimish}{github.com/Efimish} |
  \href{https://www.linkedin.com/in/efima}{linkedin.com/in/efima}
}

% Навыки
\section{Навыки}

\textbf{Языки программирования}: JavaScript/TypeScript, Rust, Python, SQL, Bash \\
\textbf{Фреймворки}: Astro, Express, NestJS \\
\textbf{Инструменты}: Git, Docker, PostgreSQL, Redis, NGINX, Caddy \\
\textbf{Остальное}: GitHub Actions

% Проекты
\section{Проекты}

% Личный веб-сайт
\textbf{Личный веб-сайт} \hfill \textit{Astro, TypeScript} \\
Личный статический веб-сайт, где можно вести блог. \\
Ссылка на веб-сайт: \href{https://efimish.github.io/}{efimish.github.io} \\
GitHub: \href{https://github.com/Efimish/efimish.github.io}{github.com/Efimish/efimish.github.io} \\

% Университетский проект вебсайт & api
\textbf{Веб-сайт для приюта} \hfill \textit{TypeScript, NestJS, Docker, PostgreSQL, Yandex.cloud} \\
Веб-сайт вымышленного приюта для домашних животных. \\
Университетский проект, созданный командой из 5 человек. \\
- Я отвечал за базу данных и API бэкенда. \\
- Я помогал подключать API к фронтенду. \\
Ссылка на веб-сайт: \href{https://ikit-group.github.io/Student-Project/}{ikit-group.github.io/Student-Project} \\
GitHub (фронт-енд): \href{https://github.com/IKIT-Group/Student-Project}{github.com/IKIT-Group/Student-Project} \\
GitHub (бэк-енд): \href{https://github.com/IKIT-Group/backend}{github.com/IKIT-Group/backend} \\

% Дискорд бот
\textbf{Дискорд бот} \hfill \textit{TypeScript, PostgreSQL, Prisma ORM} \\
Бот для Discord, который умеет играть музыку в голосовых каналах. \\
GitHub: \href{https://github.com/Efimish/Bottie}{github.com/Efimish/Bottie} \\

% Телеграм бот
\textbf{Телеграм бот} \hfill \textit{Python, PostgreSQL, SQLAlchemy} \\
Бот для Telegram, который использует ChatGPT для ответов.

% Образование
\section{Образование}

\textbf{Сибирский федеральный университет}
\hfill \textit{2023 - 2027} \\
Прикладная информатика - Бакалавриат

\end{document}
